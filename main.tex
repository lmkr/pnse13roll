\documentclass{llncs}
%%%%%%%%%%%%%%%%%%%%

%\usepackage{a4wide}
\usepackage{url}
\pagestyle{plain}
%%%%%%%%%%%%%%%%%%%%
\newcommand{\com}[1]{
        \mbox{}
       \marginpar{\hrule\footnotesize\raggedright\hspace{0pt}#1\vspace{2mm}}
}

\newcommand\concept[1]{{\em #1}}
\newcommand\ignore[1]{}


%%%%%%%%%%%%%%%%%%%%
\title{
  A Coloured Petri Net based Approach to Software Engineering for the
  TinyOS Platform
}
\author{
  Vegard Veiset \and Lars Michael Kristensen
}
\institute{
  Department of Computing, Bergen University College \\
  Email: \email{vegard.veiset@stud.hib.no,lmkr@hib.no}
}
%%%%%%%%%%%%%%%%%%%%
\begin{document}

\maketitle

\begin{abstract}

TinyOS is a widely used platform for the development of networked
embedded systems offering a programming model specifically targeting
resource constrained devices. We present a software engineering
approach where Coloured Petri Nets models can be used as a starting
point for developing protocol software for the TinyOS platform. The
approach consists of five refinement steps in which a
platform-independent CPN model is gradually refined into
platform-specific model from which code can be automatically
generated. We illustrate our approach by deriving an implementation of
the Roll routing protocol for sensor networks specified by the
Internet Engineering Task Force (IETF).

\end{abstract}

\paragraph{\textbf{Introduction.}}

Model-based software engineering and verification offers many
attractive properties in the development of flexiable and reliable
software systems. In order to fully leverage the investment into
modelling using the constructed models directly for the implementation
of the software on the platform under consideration. Coloured Petri
Nets (and Petri Nets in general) constitute a general purpose
modelling language supporting platform-independent modelling of
concurrent software systems. This means that in many cases CPN models
are too abstract to be used directly for the implementation of a
system. In order to bridge the gap between abstract and platform
independent models and the implementation of software to be deployed a
concept of pragmatics were introduced in \cite{X}. Pragmatics are
syntactical annotation that can be added to the CPN model and which is
used to direct the code generation from a specific platform. The
contribution of this paper it the development of an approach based on
the ideas of \cite{X} to develop an approach that allows code for the
TinyOS platform to be generated from CPN models. TinyOS is a platform
targetting the development of software for resource constrained
devices. Application for the TinyOS platform are implemented using the
NesC programming language which is a dialect of C. Application written
in NesC is organised into a wired set of modules each provising an
interface consists of command and events. A central aspect of the
programming model is the use of split-phase operations in order to
avoid lenghtly and time-consumsing operations that would lead to a
violation of real-time constraints.

\paragraph{\textbf{Refinement Steps and Code Generation.}}

Our approach consists of five steps starting from a platform
independent CPN model of the protocol under consideration. This CPN
model is typically constructed with the aim of specifying the
operation of the protocol and support the verification of the protocol
design. Each step consists of transformation of the CPN model that
uses the constructs of the CPN modelling language to refine the
model. Furthermore, in each step pragmatic annotation are added that
will direct the code generation after the fifth step.

\begin{description}
\item[Step 1: Component Architecture] consist of annotating CPN
  submodules and substitution transition corresponding to TinyOS
  components, and by inscribing the connected CPN arcs with pragmatics
  specifying which TinyOS interfaces they are using and providing.

\item[Step 2: Resolving Interface Conflicts] resolves interface
  conflict allowing components to use multiple instances of a single
  interface. This is done by annotating the CPN arcs with information
  provising locally unique names.

\item[Step 3: Component and Interface Signature] adds type signatures
  to components and interfaces. This is done by separating the CPN
  logic into submodules representing TinyOS events and commands, and
  by refinining colour ses CPN places to describe the interface
  signatures.

\item[Step 4: Component Classification] further refines the component
  by classifying them into four main types: timed compo- nents,
  external components, components executed at application boot, and
  general components.

\item[Step 5: Internal Component Behaviour] consist of refining the
  modeling of the individual command and events so that control flow
  and data manipulation becomes explicit and split into basic
  statement blocks.

\end{description}

After the fifth refinment step has been performed the CPN model now
includes sufficient detail to be used as a basis for automated code
generation. The automatic code generation is supported by the computer
tools that we have developed which uses the Access/CPN library to load
CPN model created with CPN tools and which uses the pragmatic
annotation of the refined model to implement a template-based code
generation approach in which the CPN model is traversed in a top-down
manner and code templates are selected according to the pragmatic
annotation.
 
To validate our approach on an industrial-sized example we have
conducted a case study using our approach and software prototype on
the Roll routing protocol developed by the IETF. The Roll protocol
allows a set of sensor nodes to form a routing tree which can be used
for communication among the sensor nodes. The formation of the routing
tree is based on the exhange of X messages among the sensor
nodes. 

% tool support

\paragraph{\textbf{Outlook}.}

At this stage a first proof-of-concept for our approach has been
established via the case study performed on the Roll protocol. This
case study will have to be complemented by additional
examples. Furthermore, at this stage the steps of the approach has
been introduced in an informal manner. Formally defining the
meta-models corresponding to each of the step would be required to
precisely specify the approach. Another direction is the application
of verification techniques at each step of the approach in order to
ensure that the key properties are preserved by the model
transformations performed at each step, and to also investigate
whether verification is appliable from the abstract to the refined
models.


\bibliographystyle{plain}
\bibliography{bib}

\end{document}
